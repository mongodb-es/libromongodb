\chapter{Operaciones avanzadas en MongoDB}

\section{Operadores relacionales o de comparación}

Los operadores relacionales son símbolos que se usan para comparar dos valores. Si el resultado de la comparación es correcto la expresión considerada es verdadera, en caso contrario es falsa.

\begin{table}[ht] 
\caption{Operadores relacionales}
\centering
\begin{tabular}{c c c c}
\hline\hline
Símbolo & Nombre & Ejemplo & Significado \\ [0.5ex] 
\hline
> & mayor que & A > B & A es mayor que B \\ 
< & menor que & A < B & A es menor que B \\
>= & mayor o igual que & A >= B & A es mayor o igual que B \\
<= & menor o igual que & A <= B & A es menor o igual que B \\ 
== & igual a & A == B & A es igual a B \\ 
!= & distinto a & A != B & A es distinto a B \\ [1ex]
\hline
\end{tabular} 
\label{table:nonlin} 
\end{table}

\subsection{Equivalencia de operadores relaciones en MongoDB}

El motor de base de datos MongoDB permite hacer operaciones sobre colecciones de datos aplicando todos los operadores relaciones de la tabla 4.1, pero con ciertas variaciones.

\begin{table}[ht] 
\caption{Tabla de equivalencias de operadores relacionales}
\centering
\begin{tabular}{c c c c}
\hline\hline
Símbolo & Nombre & Equivalencia \\ [0.5ex] 
\hline
> & mayor que & \bf{\$gt} \\ 
< & menor que & \bf{\$lt} \\
>= & mayor o igual que & \bf{\$gte} \\
<= & menor o igual que & \bf{\$lte} \\
== & igual a & \bf{:} \\ 
!= & distinto a & \bf{\$ne} \\ [1ex]
\hline
\end{tabular} 
\label{table:nonlin} 
\end{table}

La tabla 4.2 muestra claramente las equivalencias de los operadores relacionales \footnote{La equivalencia para "igual a" no existe como tal, cualquier operación donde se especifique el valor exacto luego de dos puntos se considera "igual a"}, pero existen 2 operadores más considerados relacionales en MongoDB los cuales son \textbf{\$in} y \textbf{\$nin}.

\subsection{\$gt (mayor que) | \$gte (mayor o igual que)}

Los operadores \textbf{\$gt (greater)} y \textbf{\$gte (greater or equal)} son operadores que se pueden utilizar para hacer cualquier tipo de operaciones a una coleccion de documentos, el uso es similar tanto si se utiliza en consulta, escritura o borrado de documentos. 

Un ejemplo de busqueda.

\begin{lstlisting}
    > db.personas.find({edad: {$gt: 10}})
    > db.personas.find({edad: {$gte: 10}})
\end{lstlisting}

\subsection{\$lt (menor que) | \$lte (menor o igual que)}

Los operadores \textbf{\$lt (less)} y \textbf{\$lte (less or equal)} son operadores con un uso similar a \$gt y \$gte. 

Un ejemplo de busqueda.

\begin{lstlisting}
    > db.personas.find({edad: {$lt: 10}})
    > db.personas.find({edad: {$lte: 10}})
\end{lstlisting}

\subsection{Operador de igualdad y desigualdad}

El operador de igualdad en MongoDB no se ve reflejado en simbolos como los operadores de desigualdad, sencillamente son los dos puntos seguidos al valor de la igualdad.

\begin{lstlisting}
    > db.personas.find({edad: 10})
\end{lstlisting}

Este ejemplo consulta todos los documentos que tengan un campo edad igual a 10.

En caso contrario, la desigualdad si tiene una equivalencia \$ne, que significa \textbf{not equal}, y su uso es similar a los otros operadores de mayor que, menor que, mayor igual que y menor o igual que.

\subsection{\$in (Existe en) | \$nin (No existe en)}

Existen dos operadores relacionales adicionales \textbf{\$in (exist in)} y \textbf{\$nin (do not exist in)}, que serían similar al cuantificador existencial en Matemáticas que se denota con $\exists$. \$in busca todos los documentos que coicidan con al menos uno de los valores contenido en el array de valores para la busqueda, y \$nin busca todos los documentos que no tengan campos que coincidan con los valores del array.

Ejemplo:

\begin{lstlisting}
    > db.personas.find({edad: {$in: [9, 10]}})
\end{lstlisting}

En este ejemplo MongoDB busca todos los documentos que tengan un campo edad con valor 9, 10 o ambos.

\begin{lstlisting}
    > db.personas.find({edad: {$nin: [9, 10]}})
\end{lstlisting}

En este otro ejemplo MongoDB busca todos los documentos en el cual el campo edad no coincidan con los valores 9, 10 o ambos.

\section{Operadores logicos}

